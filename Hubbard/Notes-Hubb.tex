\documentclass[10pt]{revtex4-1}

\usepackage{graphicx,bm,color}
\usepackage{epsfig}
%\usepackage{cite}
\usepackage{url}
%\usepackage[font={small,it},labelfont={bf,up}]{caption}
\usepackage{lipsum}
\usepackage{braket}
\usepackage{amsmath,amsfonts}
%\usepackage{natbib}
\usepackage{wrapfig}

\newcommand{\bfr}{{\bf r}}
\newcommand{\bfx}{{\bf x}}
\newcommand{\ud}{{\uparrow\downarrow}}
\newcommand{\du}{{\downarrow\uparrow}}
\newcommand{\uu}{{\uparrow\uparrow}}
\newcommand{\dd}{{\downarrow \downarrow}}
\newcommand{\ua}{\uparrow}
\newcommand{\da}{\downarrow}
\newcommand{\bfX}{{\bf X}}
\newcommand{\bfY}{{\bf Y}}
\newcommand{\bfA}{{\bf A}}
\newcommand{\bfB}{{\bf B}}
\newcommand{\bfm}{{\bf m}}
\newcommand{\bfk}{{\bf k}}
\newcommand{\bfq}{{\bf q}}
\newcommand{\bfG}{{\bf G}}
\newcommand{\Change}{\color{blue}}
\newcommand{\paup}{{\psi_{\alpha \ua}}}
\newcommand{\padn}{{\psi_{\alpha \da}}}
\newcommand{\pbup}{{\psi_{\beta \ua}}}
\newcommand{\pbdn}{{\psi_{\beta \da}}}

\begin{document}

\title{Non-interacting electrons on two-site lattice}

\section{Minimization Function}

Density Functional Theory (DFT) uses the Hohenberg-Kohn theorem that proved that there is a 1-to-1 correspondence between a non-interacting density and the Kohn-Sham potential for a system.
Therefore, if a given system has a non-degenerate density, there is one, and only one, potential that will create that density distribution.
Given the power of this 1-to-1 relationship, we can surmise that one can solve this relationship in either direction.
That is to say, one could solve for the density distribution of the system and then find the corresponding potential, or one could solve for the potential and then find the density that corresponds with that potential.

However, solving for the potential that gives the noninteracting density turns out to be much more difficult and nuanced.
One way to solve this problem is to self-consistently solve for the density, by using perturbation theory to find the derivative of some function which will solve for the noninteracting density.

For our work, we will use a functional that effectively takes the mean-squared difference between our target density and the density at our current step.

\begin{equation}
  S[V(\bfr)] = \int \Big (\vec{n}[V(\bfr)] - \vec{n}_{\text{t}}[V(\bfr)]\Big)^2 d\bfr
\end{equation}

We notice that this equation is positive definite and also reaches a minimum when $\vec{n} \text{ is equal to } \vec{n}_{\text{t}}$.
Such a definition of a function is ideal for use with the conjugate gradient method, which will minimize a function as long as it is derivable and positive definite.

Having defined our functional, we can take the derivative with respect to the potentials in our system.

\begin{equation}
  \nabla S \equiv \left(\frac{\delta S}{\delta V_0},\frac{\delta S}{\delta V_1},\dots,\frac{\delta S}{\delta V_N}\right)
\end{equation}

Note that the functional $S$ in the equation above still has dependence on the function $V$.
Likewise, $V$ still has spatial dependence which has been dropped for brevity.
Applying the definition of this gradient, we can find the derivative of our functional for any system.
We can write out the functional derivative as a smooth functional that depends on the continuous quantity $\bfr$.

\begin{equation}
  \frac{\delta S}{\delta V} = 2 \int (\vec{n} - \vec{n}_{\text t}) \frac{\delta \vec{n}}{\delta V}
\end{equation}

Now that we have a functional that should be differentiable and have a well-defined minimum, we can look at the actual derivative of importance in our equation.
We must first define our density.
Since we are hoping to apply this to a general system, we will write the equations in spin-dependent form.
Thus, $n$ is actually a spin-density matrix where the wavefunctions of our system are described by 2-component spinors.

\begin{eqnarray}
  \nonumber \underline{\underline{n}}(\bfr) &=&  \sum_{i=1}^N \Psi_i(\bfr)\Psi_i(\bfr)^{\dagger} \\
  \nonumber
  &=& \sum_{i=1}^N
  \begin{pmatrix}
    |\psi_{i \ua}(\bfr)|^2 & \psi_{i \ua}(\bfr) \psi_{i \da}^* (\bfr) \\
    \psi_{i \da}(\bfr) \psi_{i \ua}^* (\bfr) & |\psi_{i \da}(\bfr)|^2 \\
  \end{pmatrix} \\
  &\equiv&
  \begin{pmatrix}
    n_{\ua \ua}(\bfr) & n_{\ua \da}(\bfr) \\
    n_{\da \ua}(\bfr) & n_{\da \da}(\bfr) \\
  \end{pmatrix}
\end{eqnarray}
Here, $\Psi_i(\bfr)=
\begin{pmatrix}
  \psi_{i \ua} \\
  \psi_{i \da}
\end{pmatrix}$
and $\vec{n}(\bfr) \equiv \text{vec}\{\underline{\underline{n}}(\bfr)\}$.
From the spin-density matrix, we can find the magnetization in the x-, y-, and z-directions.
I will not explicitly write these out, but rather refer to Dr. Ullrich's DFT paper on noncollinear spin functional benchmarks for the Hubbard dimer.

\section{Potential matrix}

From Kohn-Sham equations, we can write the Kohn-Sham potential matrix as the following equation.

\begin{equation}
  \underline{\underline{v}}^{\text{KS}}(\bfr) = \sum^3_{j=0} \sigma_j V_j^{\text{KS}}(\bfr)
\end{equation}

Defining $V_j^{\text{KS}}$ as a vector made up of the potentials in the system, $B_x$, $B_y$, and $B_z$, we can find the $\underline{\underline{v}}^{\text{KS}}_j$ by carrying out our multiplication of the Pauli matrices on $V_j^{\text{KS}}$.

\begin{equation*}
  \underline{\underline{v}}^{\text{KS}}(\bfr) = \sigma_0 V_0^{\text{KS}}(\bfr) + \sigma_1 V_1^{\text{KS}}(\bfr) + \sigma_2 V_2^{\text{KS}}(\bfr) + \sigma_3 V_3^{\text{KS}}(\bfr)
\end{equation*}
\begin{equation*}
  \underline{\underline{v}}^{\text{KS}}(\bfr) =
  \begin{pmatrix}
    1 & 0 \\
    0 & 1 \\
  \end{pmatrix}
  V_0^{\text{KS}}(\bfr) +
  \begin{pmatrix}
    0 & 1 \\
    1 & 0 \\
  \end{pmatrix}
  V_1^{\text{KS}}(\bfr) +
  \begin{pmatrix}
    0 & -i \\
    i & 0 \\
  \end{pmatrix}
  V_2^{\text{KS}}(\bfr) +
  \begin{pmatrix}
    1 & 0 \\
    0 & -1 \\
  \end{pmatrix}
  V_3^{\text{KS}}(\bfr)
\end{equation*}

Then we combine them altogether, replacing our $V_j$ potentials with the actual potentials.

\begin{equation*}
  \underline{\underline{v}}^{\text{KS}}(\bfr) =
  \begin{pmatrix}
    V(\bfr) + B_z(\bfr) & B_x(\bfr) - iB_y(\bfr) \\
    B_x(\bfr) + iB_y(\bfr) & V(\bfr) - B_z(\bfr) \\
  \end{pmatrix}
\end{equation*}

Now, we need to determine the derivatives of our wavefunctions with respect to $V_j$.

\begin{equation}
   \ket{\delta \psi_n(\bfr)} = \sum_{k \neq j} \frac{\bra{\psi_k(\bfr')} \delta V(\bfr') \ket{\psi_n(\bfr')}}{E_n^{(0)}-E_k^{(0)}} \ket{\psi_k(\bfr)}
\end{equation}

Using our spinor notation, we can reorganize our equation to get our perturbation equation with equation.

\begin{equation*}
  \begin{pmatrix}
    \delta \psi_{i \ua}(\bfr) \\
    \delta \psi_{i \da}(\bfr) \\
  \end{pmatrix}
  = \sum_{j \neq i} \int d\bfr' \frac{
  \begin{pmatrix}
    \psi_{j \ua}^*(\bfr') & \psi_{j \da}^*(\bfr') \\
  \end{pmatrix}
  \underline{\underline{v}}^{\text{KS}}(\bfr')
  \begin{pmatrix}
    \psi_{i \ua}(\bfr') \\
    \psi_{i \da}(\bfr') \\
  \end{pmatrix}}
  {E_i - E_j}
  \begin{pmatrix}
    \psi_{j \ua}(\bfr) \\
    \psi_{j \da}(\bfr) \\
  \end{pmatrix}
\end{equation*}

\begin{equation*}
  \begin{pmatrix}
    \delta \psi_{i \ua}(\bfr) \\
    \delta \psi_{i \da}(\bfr) \\
  \end{pmatrix}
  = \sum_{j \neq i}
  \begin{pmatrix}
    \psi_{j \ua}(\bfr) \\
    \psi_{j \da}(\bfr) \\
  \end{pmatrix}
  \frac{1}{E_i - E_j}
  \int
  \begin{pmatrix}
    \psi_{j \ua}^*(\bfr') & \psi_{j \da}^*(\bfr') \\
  \end{pmatrix}
  \underline{\underline{v}}^{\text{KS}}(\bfr')
  \begin{pmatrix}
    \psi_{i \ua}(\bfr') \\
    \psi_{i \da}(\bfr') \\
  \end{pmatrix}
  d\bfr'
\end{equation*}

Here, we still have our perturbation equation written as a continuous function.
In order to put this into a computationally usable form, we must discretize our space.
In order to do this, we will transform our notation, where greek letters $\alpha \text{ and } \beta$ will represent eigenstates of our system, while $j$ and $k$ will represent real space coordinates - i.e. $j$ and $k$ will replace $\bfr$ and $\bfr'$.

\begin{equation}
  \delta \Psi_{\alpha}(j) \\
  = \sum_{\beta \neq \alpha} \frac{\bra{\Psi_{\beta}(k)} \delta V(k) \ket{\Psi_{\alpha}(k)}}
  {E_{\alpha} - E_{\beta}}
  \Psi_{\beta}(j)
\end{equation}
Multiplying this matrix equation out, we can explicitly define the variables we are deriving with respect to.
Additionally, we can see the summations explicitly.
\begin{multline}
  \hspace{1.2cm}
  \begin{pmatrix}
    \delta \psi_{\alpha \ua}(j) \\
    \delta \psi_{\alpha \da}(j) \\
  \end{pmatrix}
  = \sum_{\beta \neq \alpha}
  \begin{pmatrix}
    \psi_{\beta \ua}(j) \\
    \psi_{\beta \da}(j) \\
  \end{pmatrix}
  \frac{1}{E_{\alpha} - E_{\beta}} \times \\
  \sum_{k=1}^{\text{sites}} \left\{ \left([V_0(k) + V_3(k)]\psi_{\alpha \ua}(k) + [V_1(k) - i V_2(k)]\psi_{\alpha \da}(k)\Big{)}\psi_{\beta \ua}^*(k) +  \right([V_1(k) + i V_2(k)]\psi_{\alpha \ua}(k) + [V_0(k) -  V_3(k)]\psi_{\alpha \da}(k)\Big{)} \psi_{\beta \da}^*(k) \right \}
\end{multline}

Having $\delta \Psi_{\alpha}(\bfr)$, we can now look at how we can formulate our density functional derivative.
We need $\frac{\delta n(\bfr)}{\delta V_k(\bfr)}$, where $\delta V_k(\bfr)$ is the derivative with respect to the $k$-th potential as defined by the discretization of the system.
Additionally, we must calculate which derivatives will be needed for each potential/observable.
We recall that the magnetization and density of our system can be calculated from the spin density matrix/vector.
The magnetizations are written as:
\begin{equation}
  \vec{m}(\bfr) =
  \begin{pmatrix}
    m_0(\bfr) \\
    m_1(\bfr) \\
    m_2(\bfr) \\
    m_3(\bfr) \\
  \end{pmatrix}
  =
  \begin{pmatrix}
    n_{\ua \ua}(\bfr) + n_{\da \da}(\bfr) \\
    n_{\ua \da}(\bfr) + n_{\da \ua}(\bfr) \\
    i\left(n_{\ua \da}(\bfr) + n_{\da \ua}(\bfr)\right) \\
    n_{\ua \ua}(\bfr) - n_{\da \da}(\bfr) \\
  \end{pmatrix}
  .
\end{equation}

This gives us a prescription for calculating the required derivatives for each of our observables.
They can be written out explicitly as:
\begin{eqnarray}
  \nonumber
  \frac{\delta m_0(\bfr)}{\delta V_k(\bfr)} & = & \frac{\delta}{\delta V_k(\bfr)}\Big{(} n_{\ua \ua}(\bfr)+n_{\da \da}(\bfr) \Big{)} \\
  \nonumber
  \frac{\delta m_1(\bfr)}{\delta V_k(\bfr)} & = & \frac{\delta}{\delta V_k(\bfr)}\Big{(} n_{\ua \da}(\bfr)+n_{\da \ua}(\bfr) \Big{)} \\
  \nonumber
  \frac{\delta m_2(\bfr)}{\delta V_k(\bfr)} & = & i \frac{\delta}{\delta V_k(\bfr)}\Big{(} n_{\ua \da}(\bfr)+n_{\da \ua}(\bfr) \Big{)} \\
  \frac{\delta m_3(\bfr)}{\delta V_k(\bfr)} & = & \frac{\delta}{\delta V_k(\bfr)}\Big{(} n_{\ua \ua}(\bfr)-n_{\da \da}(\bfr) \Big{)}
\end{eqnarray}
Clearly, we must determine the derivative of our four matrix values in our spin density matrix.
Carrying the derivatives out, where we have omitted the potential in the derivative as we assume we are deriving with respect to $V_k(\bfr)$.
This leads to the differential density matrix $\delta n(\bfr)$.
\begin{equation}
  \delta n(\bfr) = \sum_{i=1}^N
  \begin{pmatrix}
    \delta \psi_{i \ua}(\bfr) \psi_{i \ua}^*(\bfr) + \psi_{i \ua}(\bfr) \delta \psi_{i \ua}^*(\bfr) &
    \delta \psi_{i \ua}(\bfr) \psi_{i \da}^*(\bfr) + \psi_{i \ua}(\bfr) \delta \psi_{i \da}^*(\bfr) \\
    \delta \psi_{i \da}(\bfr) \psi_{i \ua}^*(\bfr) + \psi_{i \da}(\bfr) \delta \psi_{i \ua}^*(\bfr) &
    \delta \psi_{i \da}(\bfr) \psi_{i \da}^*(\bfr) + \psi_{i \da}(\bfr) \delta \psi_{i \da}^*(\bfr)
  \end{pmatrix}
\end{equation}

After deriving the differential density matrix, we can explicitly write out the differential magnetizations that will be calculated.
\begin{equation}
  \frac{\delta \vec{m}(\bfr)}{\delta V_k(\bfr)} = \sum_{i=1}^N
  \begin{pmatrix}
    (\delta \psi_{i \ua}(\bfr) \psi_{i \ua}^*(\bfr) + \psi_{i \ua}(\bfr) \delta \psi_{i \ua}^*(\bfr)) + (\delta \psi_{i \da}(\bfr) \psi_{i \da}^*(\bfr) + \psi_{i \da}(\bfr) \delta \psi_{i \da}^*(\bfr)) \\
    (\delta \psi_{i \ua}(\bfr) \psi_{i \da}^*(\bfr) + \psi_{i \ua}(\bfr) \delta \psi_{i \da}^*(\bfr)) + (\delta \psi_{i \da}(\bfr) \psi_{i \ua}^*(\bfr) + \psi_{i \da}(\bfr) \delta \psi_{i \ua}^*(\bfr)) \\
    i\big((\delta \psi_{i \ua}(\bfr) \psi_{i \da}^*(\bfr) + \psi_{i \ua}(\bfr) \delta \psi_{i \da}^*(\bfr)) - (\delta \psi_{i \da}(\bfr) \psi_{i \ua}^*(\bfr) + \psi_{i \da}(\bfr) \delta \psi_{i \ua}^*(\bfr))\big) \\
    (\delta \psi_{i \ua}(\bfr) \psi_{i \ua}^*(\bfr) + \psi_{i \ua}(\bfr) \delta \psi_{i \ua}^*(\bfr)) - (\delta \psi_{i \da}(\bfr) \psi_{i \da}^*(\bfr) + \psi_{i \da}(\bfr) \delta \psi_{i \da}^*(\bfr)) \\
  \end{pmatrix}
\end{equation}
We can further refine our perturbation in terms of the spin-up and spin-down components of our wavefunction.
By doing this, we can explicitly write out the derivative for each wavefunction.
Doing so will provide more clarity as to which derivatives are being taken and how they are being summed together in each case.
Again, we will write the equations using notation from our discretized system ($j$ and $k$ denote spatial components, while $\alpha$ and $\beta$ denote orbitals).

\begin{eqnarray}
  \nonumber
  \frac{\partial \paup(j)}{\partial V_0(k)} &=& \sum_{\beta \neq \alpha} \pbup(j) \frac{1}{E_{\alpha}-E_{\beta}} \{ \pbup^*(k) \paup(k) + \pbdn^*(k) \padn(k) \} \\
  \nonumber
  \frac{\partial \padn(j)}{\partial V_0(k)} &=& \sum_{\beta \neq \alpha} \pbdn(j) \frac{1}{E_{\alpha}-E_{\beta}} \{ \pbup^*(k) \paup(k) + \pbdn^*(k) \padn(k) \} \\
  \nonumber
  \frac{\partial \paup(j)}{\partial V_1(k)} &=& \sum_{\beta \neq \alpha} \pbup(j) \frac{1}{E_{\alpha}-E_{\beta}} \{ \pbup^*(k) \padn(k) + \pbdn^*(k) \paup(k) \} \\
  \nonumber
   & & \hspace{2cm}\vdots  \\
  \nonumber
  \frac{\partial \paup(j)}{\partial V_2(k)} &=& \sum_{\beta \neq \alpha} \pbup(j) \frac{i}{E_{\alpha}-E_{\beta}} \{ -\pbup^*(k) \padn(k) + \pbdn^*(k) \paup(k) \} \\
  \nonumber
  & & \hspace{2cm}\vdots  \\
  \nonumber
  \frac{\partial \paup(j)}{\partial V_3(k)} &=& \sum_{\beta \neq \alpha} \pbup(j) \frac{1}{E_{\alpha}-E_{\beta}} \{ \pbup^*(k) \paup(k)) - \pbdn^*(k) \padn(k) \}
\end{eqnarray}
We note here that the summation over the perturbation step has been killed by the derivation with respect to $V_n(k)$.
By deriving with respect to our potential at point $k$, we get a Kronecker delta in our summation, $\delta_{kl}$.
Clearly, $l$ would run over all sites, while $k$ is the site on which $V_n$ is acting.

Now, let's explicitly write out what the first magnetization derivative will look like.
We recall Eq. (12), and begin by rewriting our first line:
\begin{equation*}
  \frac{\delta m_0(j)}{\delta V_n(k)} = \sum_{\alpha = 1}^N
  \left(\frac{\delta \psi_{\alpha \ua}(j)}{\delta V_n(k)} \psi_{\alpha \ua}^*(j) + \psi_{\alpha \ua}(j) \frac{\delta \psi_{\alpha \ua}^*(j)}{\delta V_n(k)}\right) + \left(\frac{\delta \psi_{\alpha \da}(j)}{\delta V_n(k)} \psi_{\alpha \da}^*(j) + \psi_{\alpha \da}(j) \frac{\delta \psi_{\alpha \da}^*(j)}{\delta V_n(k)}\right) .
\end{equation*}
Using this equation, we can now let $n=0$, giving us $\frac{\delta m_0(j)}{\delta V_0(k)}$.
Using the derivatives that we have explicitly written out above, we are left with the following.
\begin{multline*}
  \frac{\delta m_0(j)}{\delta V_n(k)}= \sum_{\alpha = 1}^N
  \Bigg( \sum_{\beta \neq \alpha} \pbup(j) \frac{1}{E_{\alpha}-E_{\beta}} \Big\{ \pbup^*(k) \paup(k) + \pbdn^*(k) \padn(k) \Big\} \paup^*(j) + \\
  \paup(j) \sum_{\beta \neq \alpha} \pbup^*(j) \frac{1}{E_{\alpha}-E_{\beta}} \Big\{ \pbup(k) \paup^*(k) + \pbdn(k) \padn^*(k) \Big\} \Bigg)
\end{multline*}
\begin{multline*}
  \hspace{1.75cm}
  + \Bigg( \sum_{\beta \neq \alpha} \pbdn(j) \frac{1}{E_{\alpha}-E_{\beta}} \Big\{ \pbup^*(k) \paup(k) + \pbdn^*(k) \padn(k) \Big\} \padn^*(j) +
  \\
  \padn(j) \sum_{\beta \neq \alpha}\pbdn^*(j) \frac{1}{E_{\alpha}-E_{\beta}} \Big\{ \pbup(k) \paup^*(k) + \pbdn(k) \padn^*(k) \Big\} \Bigg)
\end{multline*}
% ================== Bibliography =======================
%\bibliography{working-bib-corrected}
%\bibliographystyle{apsrev4-1}
%========================================================


\end{document}

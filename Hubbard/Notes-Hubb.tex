\documentclass[10pt]{revtex4-1}

\usepackage{graphicx,bm,color}
\usepackage{epsfig}
%\usepackage{cite}
\usepackage{url}
%\usepackage[font={small,it},labelfont={bf,up}]{caption}
\usepackage{lipsum}
\usepackage{braket}
\usepackage{amsmath,amsfonts}
%\usepackage{natbib}
\usepackage{wrapfig}

\newcommand{\bfr}{{\bf r}}
\newcommand{\bfx}{{\bf x}}
\newcommand{\ud}{{\uparrow\downarrow}}
\newcommand{\du}{{\downarrow\uparrow}}
\newcommand{\uu}{{\uparrow\uparrow}}
\newcommand{\dd}{{\downarrow \downarrow}}
\newcommand{\ua}{\uparrow}
\newcommand{\da}{\downarrow}
\newcommand{\bfX}{{\bf X}}
\newcommand{\bfY}{{\bf Y}}
\newcommand{\bfA}{{\bf A}}
\newcommand{\bfB}{{\bf B}}
\newcommand{\bfm}{{\bf m}}
\newcommand{\bfk}{{\bf k}}
\newcommand{\bfq}{{\bf q}}
\newcommand{\bfG}{{\bf G}}
\newcommand{\Change}{\color{blue}}

\begin{document}

\title{Non-interacting electrons on two-site lattice}

\section{Potential matrix}

From Kohn-Sham equations, we can write the Kohn-Sham potential matrix as the following equation.

\begin{equation}
  \underline{\underline{v}}^{\text{KS}}(\bfr) = \sum^3_{j=0} \sigma_j V_j^{\text{KS}}(\bfr)
\end{equation}

Defining $V_j^{\text{KS}}$ as a vector made up of the potentials in the system, $B_x$, $B_y$, and $B_z$, we can find the $\underline{\underline{v}}^{\text{KS}}_j$ by carrying out our multiplication of the Pauli matrices on $V_j^{\text{KS}}$.

\begin{equation*}
  \underline{\underline{v}}^{\text{KS}}(\bfr) = \sigma_0 V_0^{\text{KS}} + \sigma_1 V_1^{\text{KS}} + \sigma_2 V_2^{\text{KS}} + \sigma_3 V_3^{\text{KS}}
\end{equation*}
\begin{equation*}
  \underline{\underline{v}}^{\text{KS}}(\bfr) =
  \begin{pmatrix}
    1 & 0 \\
    0 & 1 \\
  \end{pmatrix}
  V_0^{\text{KS}} +
  \begin{pmatrix}
    0 & 1 \\
    1 & 0 \\
  \end{pmatrix}
  V_1^{\text{KS}} +
  \begin{pmatrix}
    0 & -i \\
    i & 0 \\
  \end{pmatrix}
  V_2^{\text{KS}} +
  \begin{pmatrix}
    1 & 0 \\
    0 & -1 \\
  \end{pmatrix}
  V_3^{\text{KS}}
\end{equation*}

Then we combine them altogether, replacing our $V_j$ potentials with the actual potentials.

\begin{equation*}
  \underline{\underline{v}}^{\text{KS}}(\bfr) =
  \begin{pmatrix}
    V + B_z & B_x - iB_y \\
    B_x + iB_y & V - B_z \\
  \end{pmatrix}
\end{equation*}

Now, we need to determine the derivatives of our wavefunctions with respect to $V_j$.

\begin{equation}
   \ket{\delta \phi_n} = \sum_{k \neq j} \frac{\bra{\phi_k} \delta V \ket{\phi_n}}{E_n^{(0)}-E_k^{(0)}} \ket{\phi_k}
\end{equation}

Using our spinor notation, we can reorganize our equation to get our perturbation equation with equation.

\begin{equation*}
  \begin{pmatrix}
    \delta \phi_{i \ua} \\
    \delta \phi_{i \da} \\
  \end{pmatrix}
  = \sum_n \sum_{j \neq i} \frac{
  \begin{pmatrix}
    \phi_{j \ua}^* & \phi_{j \da}^* \\
  \end{pmatrix}
  \underline{\underline{v}}^{\text{KS}}
  \begin{pmatrix}
    \phi_{i \ua} \\
    \phi_{i \da} \\
  \end{pmatrix}}
  {E_i - E_j}
  \begin{pmatrix}
    \phi_{j \ua} \\
    \phi_{j \da} \\
  \end{pmatrix}
\end{equation*}

\begin{equation}
  \begin{pmatrix}
    \delta \phi_{i \ua} \\
    \delta \phi_{i \da} \\
  \end{pmatrix}
  = \sum_n \sum_{j \neq i} \frac{
  \begin{pmatrix}
    \phi_{j \ua}^* & \phi_{j \da}^* \\
  \end{pmatrix}
  \begin{pmatrix}
    V + B_z & B_x - iB_y \\
    B_x + iB_y & V - B_z \\
  \end{pmatrix}
  \begin{pmatrix}
    \phi_{i \ua} \\
    \phi_{i \da} \\
  \end{pmatrix}}
  {E_i - E_j}
  \begin{pmatrix}
    \phi_{j \ua} \\
    \phi_{j \da} \\
  \end{pmatrix}
\end{equation}
Multiplying this matrix equation out, we can explicitly define the variables we are deriving with respect to.
\begin{equation}
  \begin{pmatrix}
    \delta \phi_{i \ua} \\
    \delta \phi_{i \da} \\
  \end{pmatrix}
  = \sum_{j \neq i} \bigg{\{ } \Big{(}[V_0 + V_3]\phi_{i \ua} + [V_1 + i V_2]\phi_{i \da}\Big{)}\phi_{j \ua} + \Big{(}[V_1 + i V_2]\phi_{i \ua} + [V_0 - i V_3]\phi_{i \da}\Big{)}\phi_{j \da} \bigg{\} }
  \begin{pmatrix}
    \phi_{j \ua} \\
    \phi_{j \da} \\
  \end{pmatrix}
\end{equation}
Having $\delta \phi$, we can now look at how we can formulate our density functional derivative.
We need $\frac{\delta n}{\delta v_j}$, where $\delta v_j$ is the derivative with respect to the \textit{j}-th potential as defined by the descretization of the system.
That means we have a j-length gradient of our density, with the differential density of each step being found as:

\begin{equation}
  \frac{\delta n_i}{\delta v_j} = \frac{\delta \Psi_i}{\delta v_j} \Psi_i^{\dagger} + \Psi_i \frac{\delta \Psi_i^{\dagger}}{\delta v_j}
\end{equation}

This leads to the differential density matrix $\delta n$.
\begin{equation}
  \delta n =
  \begin{pmatrix}
    \delta \psi_{\ua} \psi_{\ua}^* + \psi_{\ua} \delta \psi_{\ua}^* &
    \delta \psi_{\ua} \psi_{\da}^* + \psi_{\ua} \delta \psi_{\da}^* \\
    \delta \psi_{\da} \psi_{\ua}^* + \psi_{\da} \delta \psi_{\ua}^* &
    \delta \psi_{\da} \psi_{\da}^* + \psi_{\da} \delta \psi_{\da}^*
  \end{pmatrix}
\end{equation}

Having established the potentials and the derivative method, we can formulate the 'density' vector that we will be calculating. Each potential has a 1-to-1 correspondence with with the various density and magnetization quantities.

\begin{equation}
  \begin{pmatrix}
    n_a \\
    n_b \\
    m_{xa} \\
    m_{xb} \\
    m_{ya} \\
    m_{yb} \\
    m_{za} \\
    m_{zb} \\
  \end{pmatrix}
  \longleftrightarrow
  \begin{pmatrix}
    V_a \\
    V_b \\
    B_{xa} \\
    B_{xb} \\
    B_{ya} \\
    B_{yb} \\
    B_{za} \\
    B_{zb} \\
  \end{pmatrix}
\end{equation}

We can define $n_a$ and $n_b$ from the spin density matrix that are generated as a function of the spinor wavefunctions. Explicitly, we can define the spin-density matrix below.
\begin{eqnarray*}
  \underline{\underline{n}}(\bfr) &=& \sum_i^N \Psi_i(\bfr) \Psi_i^{\dagger}(\bfr) \\
  &\equiv&
  \begin{pmatrix}
    n_{\ua \ua}(\bfr) & n_{\ua \da}(\bfr) \\
    n_{\da \ua}(\bfr) & n_{\da \da}(\bfr) \\
  \end{pmatrix}
\end{eqnarray*}

Having defined our spin-density array, we can calculate the total density $n(\bfr) \equiv m_0(\bfr)$ where we can define the magnetization vector by using the spin-density matrix and the Pauli matrices.
\begin{equation}
  m_i(\bfr) = \text{Tr}\{\sigma_i \underline{\underline{n}}(\bfr)\}
\end{equation}
Here, $i$ runs over the $x$, $y$, and $z$ components, respectively.
The relationship then allows us to calculate the magnetization from spin-density matrix calculated from our Hamiltonian.
\begin{equation}
  \vec{m}(\bfr) =
  \begin{pmatrix}
    m_0(\bfr) \\
    m_1(\bfr) \\
    m_2(\bfr) \\
    m_3(\bfr) \\
  \end{pmatrix}
  =
  \begin{pmatrix}
    n_{\ua \ua}(\bfr) + n_{\da \da}(\bfr) \\
    n_{\ua \da}(\bfr) + n_{\da \ua}(\bfr) \\
    i(n_{\ua \da}(\bfr) - n_{\da \ua}(\bfr)) \\
    n_{\ua \ua}(\bfr) - n_{\da \da}(\bfr) \\
  \end{pmatrix}
\end{equation}

This means we now have a prescription to calculate the derivative with respect to $\delta v_j$ for all measurable values.
\begin{equation}
  \delta \vec{m}(\bfr) =
  \begin{pmatrix}
    (\delta \psi_{\ua} \psi_{\ua}^* + \psi_{\ua} \delta \psi_{\ua}^*) + (\delta \psi_{\da} \psi_{\da}^* + \psi_{\da} \delta \psi_{\da}^*) \\
    (\delta \psi_{\ua} \psi_{\da}^* + \psi_{\ua} \delta \psi_{\da}^*) + (\delta \psi_{\da} \psi_{\ua}^* + \psi_{\da} \delta \psi_{\ua}^*) \\
    i((\delta \psi_{\ua} \psi_{\da}^* + \psi_{\ua} \delta \psi_{\da}^*) - (\delta \psi_{\da} \psi_{\ua}^* + \psi_{\da} \delta \psi_{\ua}^*) \\
    (\delta \psi_{\ua} \psi_{\ua}^* + \psi_{\ua} \delta \psi_{\ua}^*) - (\delta \psi_{\da} \psi_{\da}^* + \psi_{\da} \delta \psi_{\da}^*) \\
  \end{pmatrix}
\end{equation}

% ================== Bibliography =======================
%\bibliography{working-bib-corrected}
%\bibliographystyle{apsrev4-1}
%========================================================


\end{document}
